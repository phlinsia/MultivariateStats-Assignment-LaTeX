\documentclass[12pt, a4paper, oneside]{ctexart}
\usepackage{amsmath, amsthm, amssymb, bm, color, framed, graphicx, hyperref, mathrsfs}
\usepackage[inline]{enumitem}
\usepackage{tikz}
\usepackage{pgfplots}
\usepackage{multirow, makecell}
\usepackage{colortbl}
\usepackage{booktabs}

\pgfplotsset{compat=1.16}

\title{\textbf{多元统计分析课程作业4}}
\author{陈王子\\202103150503}
\date{\today}
\linespread{1.5}
\definecolor{shadecolor}{RGB}{241, 241, 255}
\newcounter{problemname}
\newenvironment{problem}{\begin{shaded}\stepcounter{problemname}\par\noindent\textbf{题目\arabic{problemname}. }}{\end{shaded}\par}
\newenvironment{solution}{\par\noindent\textbf{解答. }}{\par}
\newenvironment{note}{\par\noindent\textbf{题目\arabic{problemname}的注记. }}{\par}

\begin{document}

\maketitle

\begin{problem}
    6.6.\\
    利用练习6.8中处理2和处理3的数据:
    处理2:$\begin{bmatrix} 3 \\ 3 \end{bmatrix}, \begin{bmatrix} 1 \\ 6 \end{bmatrix}, \begin{bmatrix} 2 \\ 3 \end{bmatrix}$\\
    处理3:$\begin{bmatrix} 2 \\ 3 \end{bmatrix}, \begin{bmatrix} 5 \\ 1 \end{bmatrix}, \begin{bmatrix} 3 \\ 1 \end{bmatrix}, \begin{bmatrix} 2 \\ 3 \end{bmatrix}$\\
    \begin{enumerate}[label=(\alph*)]
        \item 计算 $S_{\mathrm{p}}.$ 
        \item 用双样本方法以$\alpha=0.01$检验 $H_{0}:\mu_2-\mu_3=0$
        \item 构造 $\mu_{2 i}=\mu_{3 i}, i=1 , 2$ 的 99\%联合置信区间
    \end{enumerate}
\end{problem}

\begin{solution}
    \begin{enumerate}[label=(\alph*)]
        \item 
        处理2:样本均值向量 $\begin{bmatrix} 2 \\ 4 \end{bmatrix}$;
        样本协方差矩阵 $\begin{bmatrix} 1 & -\frac{3}{2} \\ -\frac{3}{2} & 3 \end{bmatrix}$\\
        处理3:样本均值向量 $\begin{bmatrix} 3 \\ 2 \end{bmatrix}$;
        样本协方差矩阵 $\begin{bmatrix} 2 & -\frac{4}{3} \\ -\frac{4}{3} & \frac{4}{3} \end{bmatrix}$\\
        $\therefore S_{\mathrm{pooled}}=\begin{bmatrix} 1.6 & -1.4 \\ -1.4 & 2 \end{bmatrix}$
        \item
        \[
        \begin{gathered}
        \begin{aligned}
            T^2&=n(\bar{x}_1-\bar{x}_2)^\top S_{\mathrm{pooled}}^{-1}(\bar{x}_1-\bar{x}_2)\\
            &=\begin{bmatrix} 2-3 & 4-2 \end{bmatrix} \left[ \left(\frac{1}{3}+\frac{1}{4}\right)\begin{bmatrix} 1.6 & -1.4 \\ -1.4 & 2 \end{bmatrix} \right]^{-1} \begin{bmatrix} 2-3 \\ 4-2 \end{bmatrix}=3.88\\
        \end{aligned}\\
        \frac{(n_1+n_2-2)p}{n_1+n_2-p-1} F_{p,n_1+n_2-p-1}(0.01)=\frac{(5)2}{3}F_{2,4}(0.01)=45\\
        \therefore T^2=3.88<45 \therefore \text{接受}H_0
        \end{gathered}
        \]
        \item 
        99\%联合置信区间为
        \[
        \begin{aligned}
            \mu_{21}-\mu_{31}&:(2-3)\pm\sqrt{45}\sqrt{\left(\frac{1}{3}+\frac{1}{4}\right)1.6}=-1\pm 6.5,\text{即}[-7.5, 5.5]\\
            \mu_{22}-\mu_{32}&:(4-2)\pm\sqrt{45}\sqrt{\left(\frac{1}{3}+\frac{1}{4}\right)2}=2\pm 7.2,\text{即}[-5.2, 9.2]
        \end{aligned}
        \]
    \end{enumerate}
\end{solution}

\begin{problem}
    6.7.\\
    利用例6.4中的电力需求数据的汇总统计数字计算 $T^{2}$ ,并检验假设 $H_0 : \;{\mu}_{1}-{\mu}_{2}={0}.$ (假定$\bf{\Sigma}_1=\bf{\Sigma}_2,\alpha=0.05$).
    另外,确定对拒绝 $H_{0}$ 起关键作用的均值分量的线性组合。
\end{problem}

\begin{solution}
    \[
    \begin{gathered}
        T^2= \begin{bmatrix} 74.4 & 201.6 \end{bmatrix} \left[ \left(\frac{1}{45}+\frac{1}{55}\right)\begin{bmatrix} 10963.7 & 21505.5 \\ 21505.5 & 63661.3 \end{bmatrix} \right]^{-1} \begin{bmatrix} 74.4 \\ 201.6 \end{bmatrix}=16.1\\
        \frac{(n_1+n_2-2)p}{n_1+n_2-p-1} F_{p,n_1+n_2-p-1}(0.05)=6.26
        \therefore T^2=16.1>6.26 \therefore \text{拒绝}H_0 \\
        \text{线性组合:} \hat{\bf{a}} \propto\left( \frac{1} {n_{1}} {\bf{S}}_{1}+\frac{1} {n_{2}} {\bf{S}}_{2} \right)^{-1} ( {\bar{x}}_{1}-{\bar{x}}_{2} )= S_{pooled}^{-1} ( {\bar{x}}_{1}-{\bar{x}}_{2} )=\begin{bmatrix} 0.0017 \\ 0.0026 \end{bmatrix}
    \end{gathered}
    \]
\end{solution}

\begin{problem}
    6.8.\\
    对三种处理收集了两种响应的观测值,观测值向量 $\begin{bmatrix} x_{1} \\ x_{2} \end{bmatrix}$ 为\\
    处理1:$\begin{bmatrix} 6 \\ 7 \end{bmatrix}, \begin{bmatrix} 5 \\ 9 \end{bmatrix}, \begin{bmatrix} 8 \\ 6 \end{bmatrix}, \begin{bmatrix} 4 \\ 9 \end{bmatrix}, \begin{bmatrix} 7 \\ 9 \end{bmatrix}$\\
    处理2:$\begin{bmatrix} 3 \\ 3 \end{bmatrix}, \begin{bmatrix} 1 \\ 6 \end{bmatrix}, \begin{bmatrix} 2 \\ 3 \end{bmatrix}$\\
    处理3:$\begin{bmatrix} 2 \\ 3 \end{bmatrix}, \begin{bmatrix} 5 \\ 1 \end{bmatrix}, \begin{bmatrix} 3 \\ 1 \end{bmatrix}, \begin{bmatrix} 2 \\ 3 \end{bmatrix}$\\
    \begin{enumerate}[label=(\alph*)]
        \item 像式(6-39)一样,把观测值分解为总平均、处理效应和残差三个部分.对每个变量构造相应的矩阵(见例6.9)
        \item 利用(a)中的信息,构造单因子 MANOVA表
        \item 计算威尔克斯 $\varLambda$ 统计量 $\varLambda^{*}$ ,并利用表6.3检验处理效应(取 $\alpha=0$ 再利用带巴特利特修正的 $\chi^{2}$ 近似重新检验上述假设[见式(6-43)],并比较这两个结论
    \end{enumerate}
\end{problem}

\begin{solution}
    \[
    \begin{gathered}
        \text{对于变量1:}\\
        \begin{bmatrix}
            6 & 5 & 8 & 4 & 7 \\
            3 & 1 & 2 &  &  \\
            2 & 5 & 3 &2 &  
        \end{bmatrix}_{\text{观测}}=
        \begin{bmatrix}
            4 & 4 & 4 & 4 & 4 \\
            4 & 4 & 4 &  &  \\
            4 & 4 & 4 & 4 &
        \end{bmatrix}_{\text{均值}}+
        \begin{bmatrix}
            2 & 2 & 2 & 2 & 2 \\
            -2 & -2 & -2 &  &  \\
            -1 & -1 & -1 & -1 &
        \end{bmatrix}_{\text{处理}}\\
        +
        \begin{bmatrix}
            0 & -1 & 2 & -2 & 1 \\
            1 & -1 & 0 &  &  \\
            -1 & 2 & 0 & -1 &
        \end{bmatrix}_{\text{残差}},
        {SS}_{\text{观测}}=246, {SS}_{\text{均值}}=192, {SS}_{\text{处理}}=36, {SS}_{\text{残差}}=18\\
        {SS}_\text{修正后总和}={SS}_{\text{观测}}-{SS}_{\text{均值}}=246-192=54\\
        \text{对于变量2:}\\
        \begin{bmatrix}
            7 & 9 & 6 & 9 & 9 \\
            3 & 6 & 3 &  &  \\
            3 & 1 & 1 & 3 &
        \end{bmatrix}_{\text{观测}}=
        \begin{bmatrix}
            5 & 5 & 5 & 5 & 5 \\
            5 & 5 & 5 &  &  \\
            5 & 5 & 5 & 5 &
        \end{bmatrix}_{\text{均值}}+
        \begin{bmatrix}
            3 & 3 & 3 & 3 & 3 \\
            -1 & -1 & -1 &  &  \\
            -3 & -3 & -3 & -3 &  
        \end{bmatrix}_{\text{处理}}\\
        +
        \begin{bmatrix}
            -1 & 1 & -2 & 1 & 1 \\
            -1 & 2 & -1 &  &  \\
            1 & -1 & -1 & 1 &
        \end{bmatrix}_{\text{残差}},
        {SS}_{\text{观测}}=402, {SS}_{\text{均值}}=300, {SS}_{\text{处理}}=84, {SS}_{\text{残差}}=18\\
        {SS}_\text{修正后总和}={SS}_{\text{观测}}-{SS}_{\text{均值}}=402-300=102\\
    \end{gathered}
    \]
为完整地给出 MANOVA表,还必须考虑它们的交叉乘积,将两变量中的数据按行相乘,可得交叉乘积的贡献
\[
\begin{gathered}
    {SCP}_\text{总}=6 \times 7 + 5 \times 9 + 8 \times 6 + 4 \times 9 + 7 \times 9 + 3 \times 3 + 1 \times 6 + 2 \times 3\\
    + 2 \times 3 + 5 \times 1 + 3 \times 1 + 2 \times 3 = 275\\
    {SCP}_\text{均值}=4 \times 5 + 4 \times 5 + \hdots + 4 \times 5  = 12 \times 4 \times 5=240\\
    {SCP}_\text{处理}=5(2 \times 3) + 3(-2 \times -1) + 4(-1 \times -3) = 48\\
    {SCP}_\text{残差}=0 \times -1 + -1 \times 1 + 2 \times -2 + \hdots -1\times 1=-13\\
    {SCP}_\text{修正后}=\text{总交叉乘积}-\text{均值交叉乘积}=35
\end{gathered}
\]
已知公式:
\[
    \begin{gathered}
    \text{总体均值向量数目/组数}g=3, \text{矩阵元素个数}\sum_{l=1}^{g}n_l =5+3+4=12\\
    B=\begin{bmatrix} {SS}_{\text{变量1处理}} & {SCP}_\text{处理} \\ {SCP}_\text{处理} & {SS}_{\text{变量2处理}} \end{bmatrix},
    W=\begin{bmatrix} {SS}_{\text{变量1残差}} & {SCP}_\text{残差} \\ {SCP}_\text{残差} & {SS}_{\text{变量2残差}} \end{bmatrix}
    \end{gathered}
\]

MANOVA表如下:

\begin{table}[htb]
    \centering
    \begin{tabular}{ccc}
    \toprule
    \rowcolor[HTML]{C0C0C0}
    变化来源  & 平方和与交叉乘积和矩阵SSP & 自由度d.f.   \\
    \midrule
    处理    & \(B=\begin{bmatrix} 36 & 48 \\ 48 & 84 \end{bmatrix} \)   & \(g-1=2\)    \\
    残差    & \(W=\begin{bmatrix} 18 & -13 \\ -13 & 18 \end{bmatrix} \)             & \(\sum_{l=1}^{g}n_l-g=9\)\\
    \midrule
    修正后总和 & \(B+W=\begin{bmatrix} 54 & 35 \\ 35 & 102 \end{bmatrix} \) & \(\sum_{l=1}^{g} n_l -1=2+9=11\)   \\
    \bottomrule
    \end{tabular}
\end{table}

\[
    \begin{gathered}
        \Lambda^\ast=\frac{|W|}{|B+W|}=\frac{18\times 18-(-13)\times(-13)}{54 \times 102 - 35 \times 35}=\frac{155}{4283}=0.0362
        \text{且} p=2,g=3\geqslant 2\\
        \therefore \text{查表6.3知威尔克斯分布} \left( \frac{ \sum n_{l} - g - 1 } {g - 1} \right) \left( \frac{ 1 - \sqrt{\Lambda^{*} } } { \sqrt{\Lambda^{*} } } \right)  \sim F_{2 ( g-1 ),  2 (  \sum_{n_l}-g-1 )}\\
        \because F_{4,16}(0.01)=4.77 < 17.02=\left(\frac{8}{2}\right)\left(\frac{1-\sqrt{0.0362}}{\sqrt{0.0362}}\right)\\
        \therefore \text{拒绝}H_0\text{,处理间的差异在}\alpha=0.01\text{的显著性水平下存在。}\\
    \end{gathered}
\]

\[
    \begin{gathered}
        \text{此外,公式(6-43)利用带巴特利特修正的}\chi^2\text{近似检验}\\
        - \left( n-1-\frac{ p+g }{2} \right) \ln \Lambda^\ast=
        - \left( 12-1-\frac{ 2+3 }{2} \right) \ln(0.0362)=28.209\\
        \text{公式(6-44)中}\chi^2_{p(g-1)}(\alpha)=\chi^2_{4}(0.01)=13.28<28.209\\
        \therefore \text{拒绝}H_0\text{,处理间的差异在}\alpha=0.01\text{的显著性水平下存在。}
    \end{gathered}
\]
\end{solution}



\begin{problem}
    6.13.\\
    无重复的双因子MANOVA:\\
    考虑以下双因子表给出的(注意:对因子水平的每一种组合中仅有一个观测值向量)响应$x_1,x_2$的观测值:


    \begin{tabular}{|cc|cccc|}
        \hline
        \multicolumn{2}{|c|}{\multirow{2}{*}{}}          & \multicolumn{4}{c|}{因子2}                                                                   \\ \cline{3-6} 
        \multicolumn{2}{|c|}{}                           & \multicolumn{1}{c|}{水平1}  & \multicolumn{1}{c|}{水平2}   & \multicolumn{1}{c|}{水平3}  & 水平4   \\ \hline
        \multicolumn{1}{|c|}{\multirow{3}{*}{因子1}} & 水平1 & \multicolumn{1}{c|}{$\begin{bmatrix} 6 \\ 8 \end{bmatrix}$}  & \multicolumn{1}{c|}{$\begin{bmatrix} 4 \\ 6 \end{bmatrix}$}   & \multicolumn{1}{c|}{$\begin{bmatrix} 8 \\ 12 \end{bmatrix}$} & $\begin{bmatrix} 2 \\ 6 \end{bmatrix}$   \\ \cline{2-6} 
        \multicolumn{1}{|c|}{}                     & 水平2 & \multicolumn{1}{c|}{$\begin{bmatrix} 3 \\ 8 \end{bmatrix}$}  & \multicolumn{1}{c|}{$\begin{bmatrix} -3 \\ 2 \end{bmatrix}$}  & \multicolumn{1}{c|}{$\begin{bmatrix} 4 \\ 3 \end{bmatrix}$}  & $\begin{bmatrix} -4 \\ 3 \end{bmatrix}$  \\ \cline{2-6} 
        \multicolumn{1}{|c|}{}                     & 水平3 & \multicolumn{1}{c|}{$\begin{bmatrix} -3 \\ 2 \end{bmatrix}$} & \multicolumn{1}{c|}{$\begin{bmatrix} -4 \\ -5 \end{bmatrix}$} & \multicolumn{1}{c|}{$\begin{bmatrix} 3 \\ -3 \end{bmatrix}$} & $\begin{bmatrix} -4 \\ -6 \end{bmatrix}$ \\ \hline
        \end{tabular}\\
    
        无重复的双因子 MANOVA模型为
        \[
            \bm{X}_{lk}=\bm{\mu}+\bm{\tau}_{l}+\bm{\beta}_{k}+e_{lk}; \quad\sum_{l=1}^{g} \bm{\tau}_{l}=\, \sum_{k=1}^{b} \bm{\beta}_{k}=0
        \]
        其中 $\bm{e}_{i k}$ 为独立的 $N_{p} ( \bm{0}, \bm{\Sigma})$ 随机向量
        \begin{enumerate}[label=(\alph*)]
            \item 与例6.9类似,将两变量中的每一个变量的观测值分解为
            \[
            x_{lk}=\bar{x}+( \bar{x}_{l\cdot}-\bar{x} )+( \bar{x}_{k\cdot}-\bar{x} )+( x_{lk}-\bar{x}_{l\cdot}+\bar{x}_{\cdot k}+\bar{x} )
            \]
            对每一个响应,这个分解会产生几个 $3 \times4$ 矩阵,这里$\bar{x}$为总平均,$\bar{x}_{l\cdot}$为因子1第 $l$ 个水平下的平均,而$\bar{x}_{\cdot k}$为因子 2第 $k$ 个水平下的平均.
            \item 矩阵的每一行看成一个串成的长向量,并计算平方和与交叉乘积和矩阵
            \item 构造 MANOVA表
            \item 检验因子1和因子2的主效应
        \end{enumerate}
\end{problem}


% \[
%     \begin{gathered}
    
%     \end{gathered}
% \]

\begin{solution}
     \[
     \begin{gathered}
        \text{对于第一个变量:}\\
        \begin{bmatrix}
            6 & 4 & 8 & 2 \\
            3 & -3 & 4 & -4 \\
            3 & -3 & 4 & -4 
        \end{bmatrix}_{\text{观测}}=
        \begin{bmatrix}
            1 & 1 & 1 & 1 \\
            1 & 1 & 1 & 1 \\
            1 & 1 & 1 & 1 
        \end{bmatrix}_{\text{均值}}+
        \begin{bmatrix}
            4 & 4 & 4 & 4 \\
            -1&-1&-1&-1 \\
            -3 & -3 & -3 & -3 
        \end{bmatrix}_{\text{因子1效果}}\\
        +\begin{bmatrix}
            1 & -2 & 4 & -3 \\
            1 & -2 & 4 & -3 \\
            1 & -2 & 4 & -3 
        \end{bmatrix}_{\text{因子2效果}}+
        \begin{bmatrix}
            0 & 1 & -1 & 0 \\
            2 & -1 & 0 & -1 \\
            -2 & 0 & 1 & 1
        \end{bmatrix}_{\text{残差}}\\
        {SS}_{\text{总}}=220, 
        {SS}_{\text{均值}}=12, 
        {SS}_{\text{因子1}}=104, 
        {SS}_{\text{因子2}}=90, 
        {SS}_{\text{残差}}=14\\
        \text{对于第二个变量:}\\
        \begin{bmatrix}
            8 & 6 & 12 & 6 \\
            8 & 2 & 3 & 3 \\
            2 & -5 & -3 & -6 
        \end{bmatrix}_{\text{观测}}=
        \begin{bmatrix}
           3 & 3 & 3 & 3 \\
              3 & 3 & 3 & 3 \\
                3 & 3 & 3 & 3
        \end{bmatrix}_{\text{均值}}+
        \begin{bmatrix}
            5 & 5 & 5 & 5 \\
            1 & 1 & 1 & 1 \\
            -6 & -6 & -6 & -6
        \end{bmatrix}_{\text{因子1效果}}\\
        +\begin{bmatrix}
            3 & -2 & 1 & -2 \\
            3 & -2 & 1 & -2 \\
            3 & -2 & 1 & -2
        \end{bmatrix}_{\text{因子2效果}}+
        \begin{bmatrix}
           -3 & 0 & 3 & 0 \\
            1 & 0 & -2 & 1 \\
            2 & 0 & -1 & -1
        \end{bmatrix}_{\text{残差}}\\
        {SS}_{\text{总}}=440, 
        {SS}_{\text{均值}}=108, 
        {SS}_{\text{因子1}}=248, 
        {SS}_{\text{因子2}}=54, 
        {SS}_{\text{残差}}=30\\
        \text{交叉相乘之和:}\\
        {SCP}_\text{总}=6 \times 8 + 4 \times 6 + \hdots+ (-4) \times (-6) = 227\\
        {SCP}_\text{均值}=1 \times 3 + 1 \times 3 + \hdots + 1 \times 3 =12 \times 1 \times 3= 36\\
        {SCP}_\text{因子1处理}=4 \times 5 + 4 \times 5 + \hdots + (-3) \times (-6) = 148\\
        {SCP}_\text{因子2处理}=1 \times 3 + (-2) \times (-2) + \hdots + (-3) \times (-2) = 51\\
        {SCP}_\text{残差}=0 \times (-3) + 1 \times 0 + \hdots + 1 \times (-1) = -8\\
        {SCP}_\text{修正后}=227-36=191,\text{验证:}227=36+148+51-8\\
    \end{gathered}
\]
已知公式:
\[
    \begin{gathered}
    \text{总体均值向量数目/组数}g=3, \text{列数}b=4,\\
    B_1=\begin{bmatrix} {SS}_{\text{变量1因子1处理}} & {SCP}_\text{因子1处理} \\ {SCP}_\text{因子1处理} & {SS}_{\text{变量2因子1处理}} \end{bmatrix},
    W=\begin{bmatrix} {SS}_{\text{变量1残差}} & {SCP}_\text{残差} \\ {SCP}_\text{残差} & {SS}_{\text{变量2残差}} \end{bmatrix}
    \end{gathered}
\]
MANOVA表如下:

\begin{table}[htb]
    \centering
    \begin{tabular}{ccc}
    \toprule
    \rowcolor[HTML]{C0C0C0}
    变化来源  & 平方和与交叉乘积和矩阵SSP & 自由度d.f.   \\
    \midrule
    因子1处理    & \(B_1=\begin{bmatrix} 104 & 148 \\ 148 & 248 \end{bmatrix} \)   & \(g-1=2\)    \\
    因子2处理    & \(B_2=\begin{bmatrix} 90 & 51 \\ 51 & 54 \end{bmatrix} \)   & \(b-1=3\)    \\
    残差    & \(W=\begin{bmatrix} 14 & -8 \\ -8 & 30 \end{bmatrix} \)             & \((g-1)(b-1)=6\)\\
    \midrule
    修正后总和 & \(B_1+B_2+W=\begin{bmatrix} 208 & 191 \\ 191 & 332 \end{bmatrix} \) & \(gb-1=11\)   \\
    \bottomrule
    \end{tabular}
\end{table}

验证因子1和因子2的主效应:

\[
    \begin{aligned}
        &- \left[ {(g-1)(b-1)-\left( {\frac{p+1-(b-1)}{2}} \right) } \right] \ln \Lambda^\ast
        =-\left[ { 6- \left( \frac{2+1-2}{2} \right) }\right] \ln{\frac{|W|}{|B_2+W|}}\\
        &=-6\ln{\frac{356}{6887}}=17.77 >\chi^2_{4}(0.05)=9.49\\ 
        &\therefore \text{结论:在}\alpha=0.05\text{的显著性水平下拒绝原假设}H_0:\tau_1=\tau_2=\tau_3=0
    \end{aligned}
\]

\[
    \begin{aligned}
        &- \left[ {(g-1)(b-1)-\left( {\frac{p+1-(g-1)}{2}} \right) } \right] \ln \Lambda^\ast
        =-\left[ { 6- \left( \frac{2+1-(4-1)}{2} \right) }\right] \ln{\frac{|W|}{|B_1+W|}}\\
        &=-5.5\ln{\frac{356}{13204}}=19.87> \chi^2_{6}(0.05)=12.59\\
        &\therefore \text{结论:在}\alpha=0.05\text{的显著性水平下拒绝原假设}H_0:\beta_1=\beta_2=\beta_3=\beta_4=0
    \end{aligned}
\]

\end{solution}

% \begin{problem}
%     5.1
%     \begin{enumerate}[label=(\alph*)]
%         \item 使用给定的数据计算检验统计量 $T^{2}$ 以检验原假设 $H_{0} \colon \mu^{\prime}= [7,11]$,
%         $$
%         \mathbf{X}= \begin{bmatrix} {2} & {12} \\ {8} & {9} \\ {6} & {9} \\ {8} & {10} \\ \end{bmatrix}
%         $$
%         \item 描述(a)部分中 $T^{2}$ 统计量的分布情况。
%         \item 结合(a)和(b)的结果,在 $\alpha=0.05$ 的显著性水平下对原假设 $H_{0}$ 进行检验。你会得出什么结论?
%     \end{enumerate}
% \end{problem}


% \begin{solution}
%     \begin{enumerate}[label=(\alph*)]
%         \item 
%         \begin{gather*}
%             \text{变量数}p=2,\text{样本大小}n=4\\
%             \bar{x}=\begin{bmatrix}
%                 \frac{2+8+6+8}{n}\\
%                 \frac{12+9+9+10}{n}
%             \end{bmatrix}
%             =\begin{bmatrix}
%                 6\\
%                 10
%             \end{bmatrix},\\
%             S_{ij} = \frac{1}{n-1} \sum_{k=1}^{n} (x_{ki} - \bar{x}_i)(x_{kj} - \bar{x}_j) ,
%             \mathbf{S}=\begin{bmatrix}
%                 8 & -\frac{10}{3}\\
%                 -\frac{10}{3} & 2
%             \end{bmatrix},\\
%             T^{2}=n(\bar{x} - \mu^\prime)^\top \mathbf{S}^{-1} (\bar{x} - \mu^\prime)=\frac{150}{11}=13.64
%         \end{gather*}
%         \item 
%         \begin{gather*}
%             \because T^{2} \sim \frac{( n-1 ) p} {( n-p )} F_{p, n-p},\\
%             \therefore T^{2} \sim 3F_{2,2}
%         \end{gather*}
%         \item 
%         \begin{gather*}
%             H_{0} \colon \mu^{\prime}= [7,11]\\
%             \because \alpha=0.05 \therefore F_{2,2}(0.05)=19\\
%             \because T^{2}=13.64<3F_{2,2}(0.05)=57\\
%             \therefore \alpha=0.05\text{时接受原假设} H_{0}
%         \end{gather*}

%       \end{enumerate}
% \end{solution}



% \begin{problem}
% 5.5.
% 样本均值向量:
% $$
% \mathbf{\bar{x}} =
% \begin{bmatrix}
% 0.564 \\
% 0.603
% \end{bmatrix}
% $$

% 样本协方差矩阵:
% $$
% \mathbf{S} =
% \begin{bmatrix}
% 0.0144 & 0.0117 \\
% 0.0117 & 0.0146
% \end{bmatrix}
% $$

% 样本协方差矩阵的逆矩阵:
% $$
% \mathbf{S}^{-1} =
% \begin{bmatrix}
% 203.018 & -163.391 \\
% -163.391 & 200.228
% \end{bmatrix}
% $$
% 在0.05的显著性水平上,对原假设 $H_{0}$ 进行检验,其中原假设为:$H_{0} \colon\boldsymbol{\mu}^{\prime} = [0.55,0.60]$。检验的结果是否与图5.1中所示的 $\boldsymbol{\mu}$ 参数95\%的置信椭圆相一致?请对此进行解释。
% \end{problem}

% \begin{solution}
%     \begin{gather*}
%         H_{0} \colon\boldsymbol{\mu}^{\prime} = [0.55,0.60],T^{2}=n(\bar{x} - \mu^\prime)^\top \mathbf{S}^{-1} (\bar{x} - \mu^\prime)=1.17\\
%         \alpha=0.05 \therefore F_{2,40}(0.05)=3.23\\
%         \because T^{2}=1.17<\frac{2(40+1)}{40},F_{2,40}(0.05)=2.05\times3.23=6.62\\   
%     \end{gather*}
%     在0.05的显著性水平上,我们不拒绝原假设$H_0$。这一结果与图中所示的95\%置信椭圆一致,因为均值向量$\boldsymbol{\mu}^\prime=[0.55,0.60]$位于该椭圆内部。
    
%     \begin{center}
%         \begin{tikzpicture}
%             \begin{axis}[
%                 xmin=0.4, xmax=0.7,
%                 ymin=0.4, ymax=0.8,
%                 axis lines=center,
%                 xlabel=$x_1$,
%                 ylabel=$x_2$,
%                 xtick={0.4,0.5,...,0.7},
%                 ytick={0.4,0.5,...,0.8},
%                 grid=major,
%             ]
%             % Draw the ellipse
%             \draw[black, fill=gray!30, rotate around={-135.2448477965646:(0.564, 0.603)}] (0.564, 0.603) ellipse (3.2373092124554926cm and 1.0582197611795cm); 
%             % 半长轴长度(major axis): 0.052910988058975*20
%             % 半短轴长度(minor axis): 0.16186546062277463*20
%             % 旋转角度(degrees): -135.2448477965646
%             % 中心坐标(center): [[0.564] [0.603]]

%             % Draw the center of the ellipse
%             \fill[black] (0.564, 0.603) circle (2pt);
%             % Draw the mu point
%             \fill[red] (0.55, 0.60) circle (2pt);
%             % Draw the dashed lines
%             \draw[dashed, red] (0.55, 0.60) -- (0.55, 0);
%             \draw[dashed, red] (0.55, 0.60) -- (0, 0.60);
%             \end{axis}
%         \end{tikzpicture}       
%     \end{center}

% \end{solution}

% \begin{problem}
%     绘制散点图,展示当显著性水平 $\alpha=0.05$ 时,对于不同的样本量 $n$(从 2 到 120,步长为 1)和不同自由度 $p$(从 1 到 60,步长为 1)条件下,变量
%     $
%     \frac{( n-1 ) p} {n-p} F_{p, n-p} ( \alpha)
%     $
%     的观测值分布情况。
% \end{problem}

% \begin{solution}
%     \begin{figure}[h]
%         \centering
%         \includegraphics[width=\textwidth]{Figure_1.png}
%         \caption{观测值分布情况}
%     \end{figure}
% \end{solution}


\end{document}