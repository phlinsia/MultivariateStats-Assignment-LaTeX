\documentclass[12pt, a4paper, oneside]{ctexart}
\usepackage{amsmath, amsthm, amssymb, bm, color, framed, graphicx, hyperref, mathrsfs}
\usepackage[inline]{enumitem}

\title{\textbf{多元统计分析期中考试}}
\author{陈王子\\202103150503}
\date{\today}
\linespread{1.5}
\definecolor{shadecolor}{RGB}{241, 241, 255}
\newcounter{problemname}
\newenvironment{problem}{\begin{shaded}\stepcounter{problemname}\par\noindent\textbf{题目\arabic{problemname}. }}{\end{shaded}\par}
\newenvironment{solution}{\par\noindent\textbf{解答. }}{\par}
\newenvironment{note}{\par\noindent\textbf{题目\arabic{problemname}的注记. }}{\par}

\begin{document}

\maketitle

\begin{problem}
    2.20\\
    已知:
    \[
    A = \begin{bmatrix}
        2 & 1 \\
        1 & 3
        \end{bmatrix}
    \]
    请计算平方根矩阵 $A^\frac{1}{2}$以及$A^{-\frac{1}{2}}$。
    然后证明下式:
    \[
    A^\frac{1}{2}A^{-\frac{1}{2}}=A^{-\frac{1}{2}}A^\frac{1}{2}=I
    \]
\end{problem}

\begin{solution}
    计算A的特征值:
    \[
        \begin{gathered}
            |A-\lambda I|=\begin{bmatrix}
                2-\lambda & 1\\
                1 & 3-\lambda
            \end{bmatrix}
            =(2-\lambda)(3-\lambda)-1=0\\
            \therefore \lambda_1=\frac{5+\sqrt{5}}{2}=3.618,
            \lambda_2=\frac{5-\sqrt{5}}{2}=1.382\\
            \text{计算矩阵A的特征值归一化特征向量对,当} Ax=\lambda x \text{时}\\
            \therefore 
                \begin{bmatrix}
                    2 & 1\\
                    1 & 3
                \end{bmatrix}
                \begin{bmatrix}
                    x_1\\
                    x_2
                \end{bmatrix}
                =3.618
                \begin{bmatrix}
                    x_1\\
                    x_2
                \end{bmatrix}\\
            \Rightarrow (x_1,x_2)=(1,1.618)\\
            \therefore e_1= (0.5257,0.8507)^T\\
            \text{同理:}\lambda_2=1.382,e_2= (0.8507,-0.5257)^T\\
            A^\frac{1}{2}
            =\sqrt{\lambda_1}e_1 e_1^T+\sqrt{\lambda_2}e_2 e_2^T
            =\begin{bmatrix}
                    1.376 & 0.325\\
                    0.325 & 1.701
                \end{bmatrix}\\
            A^{-\frac{1}{2}}
            =\frac{1}{\sqrt{\lambda_1}}e_1 e_1^T+\frac{1}{\sqrt{\lambda_2}}e_2 e_2^T
            =\begin{bmatrix}
                    0.760 & -0.145\\
                    -0.145 & 0.615
                \end{bmatrix}\\
            \text{验证则可得证:}
            A^\frac{1}{2} A^{-\frac{1}{2}}=
            \begin{bmatrix}
                1 & 0\\
                0 & 1
                \end{bmatrix}
            =A^{-\frac{1}{2}}A^\frac{1}{2}
        \end{gathered}
    \]
\end{solution}

\begin{problem}
    4.8.\\
    已知$X_1~N(0,1)$
    \begin{equation*}
    X_2=
        \begin{cases}
            -X_1,& -1\leq x_1 \leq 1\\
            X_1,&  x_1>1 \text{或} x_1<-1
        \end{cases}
    \end{equation*}
    \begin{enumerate}[label=(\alph*)]
        \item 证明$X_1~N(0,1)$
        \item 证明$X_1,X_2$不符合二元正态分布
    \end{enumerate}
\end{problem}

\begin{note}

\end{note}

\begin{solution}
    
\end{solution}

\newpage
\begin{problem}
    2.32\\
    给定了随机向量$\mathbf{X}^T=(X_1,X_2,X_3,X_4,X_5)$,均值向量$\mu^T=(2,4,-1,3,0)$,
    以及方差-协方差矩阵$\bm{\Sigma_{X}}$
    \[\bm{\Sigma_{X}}=
            \begin{bmatrix}
                4 & -1 & 0.5 & -0.5 & 0\\
                -1 & 3 & 1 & -1 & 0\\
                0.5 & 1 & 6 & 1 & -1\\
                -0.5 & -1 & 1 & 4 & 0\\
                0 & 0 & -1 & 0 & 2
            \end{bmatrix}\]
    X划分为\[\mathbf{X}=
                \begin{bmatrix}
                    X_1\\
                    X_2\\
                    \vdots\\
                    X_3\\
                    X_4\\
                    X_5
                \end{bmatrix}=
                \begin{bmatrix}
                    \mathbf{X}^{(1)}\\
                    \vdots\\
                    \mathbf{X}^{(2)}
                \end{bmatrix}\]
    又\[
        \mathbf{A}=
        \begin{bmatrix}
            1 & -1\\
            1 & 1
        \end{bmatrix},
        \mathbf{B}=
        \begin{bmatrix}
            1 & 1 & 1\\
            1 & 1 & -2
        \end{bmatrix}
        \]
    考虑$\mathbf{A}\mathbf{X}^{(1)}$和$\mathbf{B}\mathbf{X}^{(2)}$的线性组合,求:
    \begin{enumerate}[label=(\alph*)]
        \item $E(\mathbf{X}^{(1)})$
        \item $E(\mathbf{A}\mathbf{X}^{(1)})$
        \item $Cov(\mathbf{X}^{(1)})$
        \item $Cov(\mathbf{A}\mathbf{X}^{(1)})$
        \item $E(\mathbf{X}^{(2)})$
        \item $E(\mathbf{B}\mathbf{X}^{(2)})$
        \item $Cov(\mathbf{X}^{(2)})$
        \item $Cov(\mathbf{B}\mathbf{X}^{(2)})$
        \item $Cov(\mathbf{X}^{(1)},\mathbf{X}^{(2)})$
        \item $Cov(\mathbf{A}\mathbf{X}^{(1)},\mathbf{B}\mathbf{X}^{(2)})$
    \end{enumerate}
\end{problem}

\begin{solution}
    \begin{enumerate}[label=(\alph*)]
        \item 
        \[
            E(\mathbf{X}^{(1)})=\mu^{(1)}
            =\begin{bmatrix}
                2\\
                4
            \end{bmatrix}
        \]
        \item 
        \[
            \mathbf{A}\mu^{(1)}=
            \begin{bmatrix}
                1 & -1\\
                1 & 1
            \end{bmatrix}
            \begin{bmatrix}
                2\\
                4
            \end{bmatrix}=
            \begin{bmatrix}
                -2\\
                6
            \end{bmatrix}
        \]
        \item
        \[
            Cov(\mathbf{X}^{(1)})=\bm{\Sigma_{11}}=
            \begin{bmatrix}
                4 & -1 \\
                -1 & 3
            \end{bmatrix}
        \] 
        \item 
        \[
            Cov(\mathbf{A}\mathbf{X}^{(1)})=\mathbf{A}\bm{\Sigma_{11}}\mathbf{A^T}=
            \begin{bmatrix}
                1 & -1\\
                1 & 1
            \end{bmatrix}
            \begin{bmatrix}
                4 & -1 \\
                -1 & 3
            \end{bmatrix}
            \begin{bmatrix}
                1 & 1\\
                -1 & 1
            \end{bmatrix}
            =\begin{bmatrix}
                9 & 1 \\
                1 & 5
            \end{bmatrix}
        \]
        \item 
        \[
            E(\mathbf{X}^{(2)})=\mu^{(2)}=
            \begin{bmatrix}
                -1\\
                0\\
                3
            \end{bmatrix}
        \]
        \item 
        \[
            \mathbf{B}\mu^{(2)}=
            \begin{bmatrix}
                1 & 1 & 1\\
                1 & 1 & -2
            \end{bmatrix}
            \begin{bmatrix}
                -1\\
                0\\
                3
            \end{bmatrix}
            =
            \begin{bmatrix}
                2\\
                -7
            \end{bmatrix}
        \]
        \item 
        \[
            Cov(\mathbf{X}^{(2)})=\bm{\Sigma_{22}}=
            \begin{bmatrix}
                6 & 1 & -1 \\
                1 & 4 & 0\\
                -1 & 0 & 2
            \end{bmatrix}
        \] 
        \item 
        \[
            Cov(\mathbf{B}\mathbf{X}^{(2)})=\mathbf{B}\bm{\Sigma_{22}}\mathbf{B^T}=
            \begin{bmatrix}
                1 & 1 & 1\\
                1 & 1 & -2
            \end{bmatrix}
            \begin{bmatrix}
                6 & 1 & -1 \\
                1 & 4 & 0\\
                -1 & 0 & 2
            \end{bmatrix}
            \begin{bmatrix}
                1 & 1 \\
                1 & 1\\
                1 & -2
            \end{bmatrix}=
            \begin{bmatrix}
                12 & 9\\
                9 & 24
            \end{bmatrix}
        \]
        \item 
        \[
            Cov(\mathbf{X}^{(1)},\mathbf{X}^{(2)})=
            \begin{bmatrix}
                \frac{1}{2} & -\frac{1}{2} & 0\\
                1 & -1 & 0
            \end{bmatrix}
        \]
        \item 
        \[
            Cov(\mathbf{A}\mathbf{X}^{(1)},\mathbf{B}\mathbf{X}^{(2)})=\mathbf{A}\bm{\Sigma_{12}}\mathbf{B^T}\\
            =\begin{bmatrix}
                1 & 1 \\
                1 & 1\\
                1 & -2
            \end{bmatrix}
            \begin{bmatrix}
                \frac{1}{2} & -\frac{1}{2} & 0\\
                1 & -1 & 0
            \end{bmatrix}
            \begin{bmatrix}
                1 & 1 \\
                1 & 1\\
                1 & -2
            \end{bmatrix}=
            \begin{bmatrix}
                0 & 0\\
                0 & 0
            \end{bmatrix}
        \]
    \end{enumerate}
\end{solution}

\begin{problem}
    4.18.\\
    试根据来自二维正态总体的随机样本
        $$
        X=\begin{bmatrix} {3} & {6} \\ {4} & {4} \\ {5} & {7} \\ {4} & {7} \\ \end{bmatrix} 
        $$
    求 $2 \times1$ 均值向量$\mu$和 $2 \times2$ 协方差矩阵$\Sigma$的极大似然估计
\end{problem}

\begin{solution}
    根据 4.11 结果, $\mu$ 和 $\Sigma$ 的最大似然估计分别为 $\hat{\mu}=\bar{x}=[4,6]^{\prime}$
    \begin{equation*}
        \begin{aligned}
        &\frac{1}{n}\sum_{j=1}^{n}\left(x_{j}-\bar{x}\right)\left(x_{j}-\bar{x}\right)^{\prime} \\ 
            &= \frac{1}{4} 
            \left\{
                \left(\begin{bmatrix}3\\6\end{bmatrix}-\begin{bmatrix}4\\6\end{bmatrix}\right)\left(\begin{bmatrix}3\\6\end{bmatrix}-\begin{bmatrix}4\\6\end{bmatrix}\right)^{\prime}+
                \left(\begin{bmatrix}4\\4\end{bmatrix}-\begin{bmatrix}4\\6\end{bmatrix}\right)\left(\begin{bmatrix}4\\4\end{bmatrix}-\begin{bmatrix}4\\6\end{bmatrix}\right)^{\prime}\right.\\
            &\left.+\left(\begin{bmatrix}5\\7\end{bmatrix}-\begin{bmatrix}4\\6\end{bmatrix}\right)\left(\begin{bmatrix}5\\7\end{bmatrix}-\begin{bmatrix}4\\6\end{bmatrix}\right)^{\prime}+
                \left(\begin{bmatrix}4\\7\end{bmatrix}-\begin{bmatrix}4\\6\end{bmatrix}\right)\left(\begin{bmatrix}4\\7\end{bmatrix}-\begin{bmatrix}4\\6\end{bmatrix}\right)^{\prime}
            \right\}\\
                &=\frac{1}{4}
                \left( \begin{bmatrix}-1\\0\end{bmatrix} \begin{bmatrix}-1&0\end{bmatrix} +
                         \begin{bmatrix}0\\-2\end{bmatrix} \begin{bmatrix}0&-2\end{bmatrix} + 
                         \begin{bmatrix}1\\1\end{bmatrix} \begin{bmatrix}1&1\end{bmatrix} + 
                         \begin{bmatrix}0\\1\end{bmatrix} \begin{bmatrix}0&1\end{bmatrix} 
                \right)\\
            &=\frac{1}{4}\begin{bmatrix} 
                            {2}&{1}\\
                            {1}&{6}
                        \end{bmatrix}
        \end{aligned}
    \end{equation*}
\end{solution}


\begin{problem}
    5.1
    \begin{enumerate}[label=(\alph*)]
        \item 使用给定的数据计算检验统计量 $T^{2}$ 以检验原假设 $H_{0} \colon \mu^{\prime}= [7,11]$,
        $$
        \mathbf{X}= \begin{bmatrix} 
                        {2} & {12} \\ 
                        {8} & {9} \\ 
                        {6} & {9} \\ 
                        {8} & {10} \\ 
                    \end{bmatrix}
        $$
        \item 描述(a)部分中 $T^{2}$ 统计量的分布情况。
        \item 结合(a)和(b)的结果,在 $\alpha=0.05$ 的显著性水平下对原假设 $H_{0}$ 进行检验。你会得出什么结论?
    \end{enumerate}
\end{problem}


\begin{solution}
    \begin{enumerate}[label=(\alph*)]
        \item 
        \begin{gather*}
            \text{变量数}p=2,\text{样本大小}n=4\\
            \bar{x}=\begin{bmatrix}
                \frac{2+8+6+8}{n}\\
                \frac{12+9+9+10}{n}
            \end{bmatrix}
            =\begin{bmatrix}
                6\\
                10
            \end{bmatrix},\\
            S_{ij} = \frac{1}{n-1} \sum_{k=1}^{n} (x_{ki} - \bar{x}_i)(x_{kj} - \bar{x}_j) ,
            \mathbf{S}=\begin{bmatrix}
                8 & -\frac{10}{3}\\
                -\frac{10}{3} & 2
            \end{bmatrix},\\
            T^{2}=n(\bar{x} - \mu^\prime)^\top \mathbf{S}^{-1} (\bar{x} - \mu^\prime)=\frac{150}{11}=13.64
        \end{gather*}
        \item 
        \begin{gather*}
            \because T^{2} \sim \frac{( n-1 ) p} {( n-p )} F_{p, n-p},\\
            \therefore T^{2} \sim 3F_{2,2}
        \end{gather*}
        \item 
        \begin{gather*}
            H_{0} \colon \mu^{\prime}= [7,11]\\
            \because \alpha=0.05 \therefore F_{2,2}(0.05)=19\\
            \because T^{2}=13.64<3F_{2,2}(0.05)=57\\
            \therefore \alpha=0.05\text{时接受原假设} H_{0}
        \end{gather*}

      \end{enumerate}
\end{solution}

\begin{problem}
    4.11.\\
    A是方阵,请证明:
    \begin{equation*}
        \begin{aligned}
            |A|&=|A_{22}||A_{11}-A_{12}A_{22}^{-1}A_{21}| \quad (|A_{22}|\neq 0)\\
            &=|A_{11}||A_{22}-A_{21}A_{11}^{-1}A_{12}| \quad (|A_{11}|\neq 0)
        \end{aligned}
    \end{equation*}
\end{problem}

\begin{note}
    将A分块并证明:
\end{note}

\begin{problem}
    
    
\end{problem}






\end{document}