\documentclass[12pt, a4paper, oneside]{ctexart}
\usepackage{amsmath, amsthm, amssymb, bm, color, framed, graphicx, hyperref, mathrsfs}
\usepackage[inline]{enumitem}

\title{\textbf{多元统计分析课程作业1}}
\author{Phlins}
\date{\today}
\linespread{1.5}
\definecolor{shadecolor}{RGB}{241, 241, 255}
\newcounter{problemname}
\newenvironment{problem}{\begin{shaded}\stepcounter{problemname}\par\noindent\textbf{题目\arabic{problemname}. }}{\end{shaded}\par}
\newenvironment{solution}{\par\noindent\textbf{解答. }}{\par}
\newenvironment{note}{\par\noindent\textbf{题目\arabic{problemname}的注记. }}{\par}

\begin{document}

\maketitle

\begin{problem}
    考虑点$(x_1,x_2)$的集合,对于$c^2=1$和$c^2=4$从原点到该点的距离由下式给出:
    $$
    c^2=4x_1^2+3x_2^2-2\sqrt{2}x_1x_2
    $$
确定常数距离椭圆的长短轴及其相应的长度,画其草图并注明位置。当$c^2$增大时会怎么样?
\end{problem}

\begin{solution}

改变上式形式,
$$
    \because c^2=x'Ax\\
\therefore a_{11}=4,a_{22}=3,a_{12}=a_{21}=-\sqrt{2}
\therefore A = \begin{bmatrix}
4 & -\sqrt{2} \\
\sqrt{2} & 3
\end{bmatrix}
$$

计算A的特征值:

$$
|A-\lambda I|=\begin{bmatrix}
4-\lambda & -\sqrt{2}\\
-\sqrt{2} & 3-\lambda
\end{bmatrix}=(4-\lambda)(3-\lambda)-2=0\\
\therefore \lambda_1=5,\lambda_2=2
$$

计算矩阵A的特征值归一化特征向量对:

当$Ax=\lambda_1 x$时,
$$
\therefore 
\begin{bmatrix}
4-\lambda & -\sqrt{2}\\
-\sqrt{2} & 3-\lambda
\end{bmatrix}
\begin{bmatrix}
x_1\\
x_2
\end{bmatrix}=5
\begin{bmatrix}
x_1\\
x_2
\end{bmatrix}\\
\Rightarrow x_1=-\sqrt{2}x_2\\
\therefore e_1= \frac{1}{\sqrt{3}}\begin{bmatrix}
\sqrt{2}\\
-1
\end{bmatrix}\\
$$

当$Ax=\lambda_2 x$时,
$$
\therefore 
\begin{bmatrix}
4-\lambda & -\sqrt{2}\\
-\sqrt{2} & 3-\lambda
\end{bmatrix}
\begin{bmatrix}
x_1\\
x_2
\end{bmatrix}=2
\begin{bmatrix}
x_1\\
x_2
\end{bmatrix}\\
\Rightarrow x_2=\sqrt{2}x_1\\
\therefore e_2= \frac{1}{\sqrt{3}}\begin{bmatrix}
1\\
\sqrt{2}
\end{bmatrix}\\
$$

当$c^2=1$时,沿向量$e_1,e_2$的常距椭圆的半长度(长轴和短轴)为:
$$
\frac{c}{\sqrt{\lambda_1}}=\frac{1}{\sqrt{5}}=0.447,\frac{c}{\sqrt{\lambda_2}}=\frac{1}{\sqrt{2}}=0.707
$$

当$c^2=4$时,沿向量$e_1,e_2$的常距椭圆的半长度(长轴和短轴)为:
$$
\frac{c}{\sqrt{\lambda_1}}=\frac{2}{\sqrt{5}}=0.894,\frac{c}{\sqrt{\lambda_2}}=\frac{2}{\sqrt{2}}=1.414
$$

结论是:当c增大时,常距椭圆的半长度(长轴和短轴)以c的比例同时增大。


\end{solution}

\begin{note}
    第一题的图就不画了。
    $$
    e=\frac{1}{\sqrt{x_1^2+x_2^2}}\begin{bmatrix}
        x_1\\
        x_2\\
    \end{bmatrix}
    $$
\end{note}


\begin{problem}
已知:
$$
A = \begin{bmatrix}
2 & 1 \\
1 & 3
\end{bmatrix}
$$
请计算平方根矩阵 $A^\frac{1}{2}$以及$A^{-\frac{1}{2}}$。
然后证明下式:
$$
A^\frac{1}{2}A^{-\frac{1}{2}}=A^{-\frac{1}{2}}A^\frac{1}{2}=I
$$
\end{problem}

\begin{solution}
    计算A的特征值:
    $$
    |A-\lambda I|=\begin{bmatrix}
    2-\lambda & 1\\
    1 & 3-\lambda
    \end{bmatrix}=(2-\lambda)(3-\lambda)-1=0\\
    $$

    $$
    \therefore \lambda_1=\frac{5+\sqrt{5}}{2}=3.618,\lambda_2=\frac{5-\sqrt{5}}{2}=1.382
    $$
    计算矩阵A的特征值归一化特征向量对,当$Ax=\lambda x$时
    $$
    \therefore 
    \begin{bmatrix}
    2 & 1\\
    1 & 3
    \end{bmatrix}
    \begin{bmatrix}
    x_1\\
    x_2
    \end{bmatrix}=3.618
    \begin{bmatrix}
    x_1\\
    x_2
    \end{bmatrix}\\
    \Rightarrow (x_1,x_2)=(1,1.618)\\
    \therefore e_1= (0.5257,0.8507)^T\\
    $$
    $$
    \text{同理:}\lambda_2=1.382,e_2= (0.8507,-0.5257)^T
    $$
    
    $$
    A^\frac{1}{2}=\sqrt{\lambda_1}e_1 e_1^T+\sqrt{\lambda_2}e_2 e_2^T=
    \begin{bmatrix}
        1.376 & 0.325\\
        0.325 & 1.701
        \end{bmatrix}
    $$

    $$
    A^{-\frac{1}{2}}=\frac{1}{\sqrt{\lambda_1}}e_1 e_1^T+\frac{1}{\sqrt{\lambda_2}}e_2 e_2^T=
    \begin{bmatrix}
        0.760 & -0.145\\
        -0.145 & 0.615
        \end{bmatrix}
    $$
    验证则可得证:
    $$
    A^\frac{1}{2} A^{-\frac{1}{2}}=
    \begin{bmatrix}
        1 & 0\\
        0 & 1
        \end{bmatrix}
    =A^{-\frac{1}{2}}A^\frac{1}{2}
    $$

\end{solution}


\begin{problem}
已知:
$$
A = \begin{bmatrix}
4 & 8 & 8 \\
3 & 6 & -9
\end{bmatrix}
$$
\begin{enumerate}[label=(\alph*)]
    \item 计算$AA'$及其特征值和特征向量
    \item 计算$A'A$及其特征值和特征向量,(b)的结果和(a)的结果是否一致?
    \item 求出$A$的奇异值分解。
\end{enumerate}
\end{problem}


\begin{solution}
    $$
    AA^T=
    \begin{bmatrix}
        4 & 8 & 8 \\
        3 & 6 & -9
        \end{bmatrix}
        \begin{bmatrix}
            4 & 3 \\
            8 & 6\\
            8&-9
            \end{bmatrix}
            =    \begin{bmatrix}
                144 & -12\\
                -12 & 126
                \end{bmatrix}
    $$
    $$
    0=|A-\lambda I|=(144-\lambda)(126-\lambda)-(12)^2=(150-\lambda)(120-\lambda)
    $$
    $$
    \therefore \lambda_1=150,\lambda_2=120
    $$

    当$Ax=\lambda_1 x$时,
$$
\begin{bmatrix}
144 & -12\\
-12 & 126
\end{bmatrix}
\begin{bmatrix}
x_1\\
x_2
\end{bmatrix}=150
\begin{bmatrix}
x_1\\
x_2
\end{bmatrix},\\
\begin{cases}
    &144x_1-12x_2=150x_1\\
    &-12x_1+126x_2=150x_2
\end{cases},\\
x_1=-2x_2
$$
可得(a)
$$
\therefore e_1= \frac{1}{\sqrt{5}}\begin{bmatrix}
    2\\
    -1
    \end{bmatrix}
\text{同理}\lambda_2=120\text{时},e_2= \frac{1}{\sqrt{5}}\begin{bmatrix}
    1\\
    2
    \end{bmatrix}
$$

$$
A^TA=
    \begin{bmatrix}
        4 & 3 \\
        8 & 6\\
        8&-9
        \end{bmatrix}
        \begin{bmatrix}
            4 & 8 & 8 \\
            3 & 6 & -9
            \end{bmatrix}
        =            \begin{bmatrix}
            25 & 50 & 5 \\
            50 & 100 & 10\\
            5 & 10 & 145
            \end{bmatrix}
$$
$$
0=|A-\lambda I|=
\begin{bmatrix}
    25-\lambda & 50 & 5 \\
    50 & 100-\lambda & 10\\
    5 & 10 & 145-\lambda
    \end{bmatrix}=0
$$
$$
\therefore \text{解得} \lambda_1=150,\lambda_2=120,\lambda_3=0
$$
当$Ax=\lambda_1 x$时,
$$
\begin{bmatrix}
    25 & 50 & 5 \\
    50 & 100 & 10\\
    5 & 10 & 145
    \end{bmatrix}
\begin{bmatrix}
x_1\\
x_2\\
x_3
\end{bmatrix}=150
\begin{bmatrix}
x_1\\
x_2\\
x_3
\end{bmatrix},\\
\begin{cases}
    &-120x_1+60x_2=0\\
    &-25x_1+5x_3=0
\end{cases},\\
e_1=\frac{1}{\sqrt{30}}\begin{bmatrix}
    1\\
    2\\
    5
    \end{bmatrix}
$$
同理可得(b)
$$
e_2=\frac{1}{\sqrt{6}}\begin{bmatrix}
    1\\
    2\\
    -1
    \end{bmatrix},
    e_3=\frac{1}{\sqrt{5}}\begin{bmatrix}
        2\\
        -1\\
        0
        \end{bmatrix}
$$
(c)根据$A=\sqrt{\lambda_1}e_1 e_1^T+\sqrt{\lambda_2}e_2 e_2^T$
$$
A=   \begin{bmatrix}
    4 & 8 & 8 \\
    3 & 6 & -9
    \end{bmatrix}=
    \sqrt{150}
    \frac{1}{\sqrt{5}}\begin{bmatrix}
        2\\
        -1
        \end{bmatrix}
    \frac{1}{\sqrt{30}}\begin{bmatrix}
        1\\
        2\\
        5
        \end{bmatrix}+
        \sqrt{120}
    \frac{1}{\sqrt{5}}\begin{bmatrix}
        1\\
        2
        \end{bmatrix}
    \frac{1}{\sqrt{6}}\begin{bmatrix}
        1\\
        2\\
        -1
        \end{bmatrix}
$$


\end{solution}



\begin{problem}
    设$X$有协方差矩阵
$$
A = \begin{bmatrix}
25 & -2 & 4 \\
-2 & 4 & 1 \\
4 & 1 & 9
\end{bmatrix}
$$
\begin{enumerate}[label=(\alph*)]
\item 确定$\rho$和$V^\frac{1}{2}$
\item 用矩阵乘法验证关系式:
  $$
  V^\frac{1}{2} \rho V^\frac{1}{2}=\Sigma
  $$
\end{enumerate}
\end{problem}

\begin{solution}
$$
\Sigma=Cov(x)=\begin{bmatrix}
    25 & -2 &4\\
    -2&4&1\\
    4&1&9
\end{bmatrix}
$$
$$
\therefore \sigma_{11}=25,\sigma_{22}=4,\sigma_{33}=9,
\sqrt{\sigma_{11}}=5,\sqrt{\sigma_{22}}=2,\sqrt{\sigma_{33}}=3
$$
已知总体相关系数$\rho_{ik}=\frac{\sigma_{ik}}{\sqrt{\sigma_{ii}}\sqrt{\sigma_{kk}}}$
$$
\rho=\begin{bmatrix}
    \frac{25}{5\cdot 5} & \frac{-2}{2\cdot 5} & \frac{4}{3\cdot 5} \\
    \frac{-2}{2\cdot 5} & \frac{4}{2\cdot 2} & \frac{1}{3\cdot 2} \\
    \frac{4}{5\cdot 3} & \frac{1}{2\cdot 3} & \frac{9}{3\cdot 3} 
\end{bmatrix}=\begin{bmatrix}
    1 & -\frac{1}{5} & \frac{4}{15}\\
    -\frac{1}{5} & 1 & \frac{1}{6}\\
    \frac{4}{15} & \frac{1}{6} & 1
\end{bmatrix}
$$
已知$p\times p$标准差矩阵$V^{\frac{1}{2}}=\begin{bmatrix}
    \sqrt{\sigma_{11}} & 0 & \cdots & 0\\
    0 & \sqrt{\sigma_{22}} & \cdots & 0\\
    \vdots & \vdots & \ddots & \vdots \\
    0 & 0 & \cdots &\sqrt{\sigma_{pp}}\\
\end{bmatrix}
$
$$
V^{\frac{1}{2}}=\begin{bmatrix}
    5 & 0 & 0\\
    0 & 2 & 0\\
    0 & 0 & 3
\end{bmatrix}
$$
$$
V^\frac{1}{2} \rho V^\frac{1}{2}=\begin{bmatrix}
    5 & 0 & 0\\
    0 & 2 & 0\\
    0 & 0 & 3
\end{bmatrix}
\begin{bmatrix}
    1 & -\frac{1}{5} & \frac{4}{15}\\
    -\frac{1}{5} & 1 & \frac{1}{6}\\
    \frac{4}{15} & \frac{1}{6} & 1
\end{bmatrix}
\begin{bmatrix}
    5 & 0 & 0\\
    0 & 2 & 0\\
    0 & 0 & 3
\end{bmatrix}=
\begin{bmatrix}
    25 & -2 & 4 \\
    -2 & 4 & 1 \\
    4 & 1 & 9
    \end{bmatrix}=\Sigma \text{得证}
$$
\end{solution}
\begin{note}
同理
$\rho=(V^\frac{1}{2})^{-1}\Sigma(V^\frac{1}{2})^{-1}$
\end{note}

\end{document}