\documentclass[12pt, a4paper, oneside]{ctexart}
\usepackage{amsmath, amsthm, amssymb, bm, color, framed, graphicx, hyperref, mathrsfs}
\usepackage[inline]{enumitem}

\title{\textbf{多元统计分析期末考试}}
\author{Phlinsia}
\date{\today}
\linespread{1.5}
\definecolor{shadecolor}{RGB}{241, 241, 255}
\newcounter{problemname}
\newenvironment{problem}{\begin{shaded}\stepcounter{problemname}\par\noindent\textbf{题目\arabic{problemname}. }}{\end{shaded}\par}
\newenvironment{solution}{\par\noindent\textbf{解答. }}{\par}
\newenvironment{note}{\par\noindent\textbf{题目\arabic{problemname}的注记. }}{\par}

\begin{document}

\maketitle
\newpage

\begin{problem}
    2.32(改)\\
    给定了随机向量$\mathbf{X}^\top=(X_1,X_2,X_3,X_4)$,均值向量$\mu^\top=(3,2,-2,0)$,
    以及方差-协方差矩阵$\bm{\Sigma_{X}}$
    \[\bm{\Sigma_{X}}=
            \begin{bmatrix}
                3 & 0 & 0 & 0 \\
                0 & 3 & 0 & 0 \\
                0 & 0 & 3 & 0 \\
                0 & 0 & 0 & 3
            \end{bmatrix}\]
    又\[
        \mathbf{A}=
        \begin{bmatrix}
            1 & -1 & 0 & 0\\
            1 & 1 & -2 & 0\\
            1 & 1 & 1 & -3\\
        \end{bmatrix}
        \]
    求 \( AX \) 的均值 \( E(AX) \) 和协方差矩阵 \( Cov(AX) \)。
\end{problem}

\begin{solution}
    \begin{enumerate}[label=(\alph*)]
        \item 
        \[ 
            \begin{gathered}
                E(\mathbf{X})=\mu\\
                E(\mathbf{A}\mathbf{X})=\mathbf{A}\mu=\begin{bmatrix}
                    1 & -1 & 0 & 0\\
                    1 & 1 & -2 & 0\\
                    1 & 1 & 1 & -3\\
                \end{bmatrix} \begin{bmatrix}
                    3\\
                    2\\
                    -2\\
                    0
                \end{bmatrix}=\begin{bmatrix}
                    1\\
                    9\\
                    3
                \end{bmatrix}
            \end{gathered}
        \]
        \item 
        \[
            \begin{gathered}
                Cov(\mathbf{X})=\bm{\Sigma_{X}}\\
                Cov(\mathbf{A}\mathbf{X})=\mathbf{A}\bm{\Sigma_{X}}\mathbf{A^\top}=
                \begin{bmatrix}
                    1 & -1 & 0 & 0\\
                    1 & 1 & -2 & 0\\
                    1 & 1 & 1 & -3\\
                \end{bmatrix}
                \begin{bmatrix}
                    3 & 0 & 0 & 0 \\
                    0 & 3 & 0 & 0 \\
                    0 & 0 & 3 & 0 \\
                    0 & 0 & 0 & 3
                \end{bmatrix}
                \begin{bmatrix}
                    1 & 1 & 1\\
                    -1 & 1 & 1\\
                    0 & -2 & 1\\
                    0 & 0 & -3
                \end{bmatrix}=
                \begin{bmatrix}
                    6 & 0 & 0\\
                    0 & 18 & 0\\
                    0 & 0 & 36
                \end{bmatrix}
            \end{gathered}
        \]
    \end{enumerate}
\end{solution}


\newpage
\begin{problem}
    5.1(改)
    \[
        \mathbf{X}= \begin{bmatrix} 
                        {-5} & {6} \\ 
                        {1} & {3} \\ 
                        {-1} & {3} \\ 
                        {1} & {4} \\ 
                    \end{bmatrix}
        \]
        在 $\alpha=0.05$ 的显著性水平下检验原假设 $H_{0}:\mu^{\prime}= [0,5]^\top$ 
\end{problem}


\begin{solution}
        \[
        \begin{gathered}
            \text{变量数}p=2,\text{样本大小}n=4,
            \bar{x}=\begin{bmatrix}
                \frac{-5+1+-1+1}{n}\\
                \frac{6+3+3+4}{n}
            \end{bmatrix}
            =\begin{bmatrix}
                -1\\
                4
            \end{bmatrix},\\
            \because
            \mathbf{S}_{ij} = \frac{1}{n-1} \sum_{k=1}^{n} (x_{ki} - \bar{x}_i)(x_{kj} - \bar{x}_j) ,\\
            s_{11}=\frac{1}{3}((-5--1)^2+(1--1)^2+(-1--1)^2+(1--1)^2)=8,\\
            s_{12}=s_{12}=\frac{1}{3}((-5--1)(6-4)+(1--1)(3-4)\\+(-1--1)(3-4)+(1--1)(4-4))=-\frac{10}{3},\\
            s_{22}=\frac{1}{3}((6-4)^2+(3-4)^2+(3-4)^2+(4-4)^2)=2,\\
            \therefore 
            \mathbf{S}=\begin{bmatrix}
                8 & -\frac{10}{3}\\
                -\frac{10}{3} & 2
            \end{bmatrix},\text{又}
            \begin{bmatrix}
                a & b\\
                b & c
            \end{bmatrix}^{-1}=\frac{1}{ac-b^{2}}\begin{bmatrix}
                c & -b\\
                -b & a
            \end{bmatrix}
        \end{gathered}
    \]
    \[
        \begin{gathered}
            T^{2}=n(\bar{x} - \mu^\prime)^\top \mathbf{S}^{-1} (\bar{x} - \mu^\prime)=\\
            4\begin{bmatrix}
                -1-0 & 4-5
            \end{bmatrix}  \frac{9}{44} \cdot \frac{1}{3}
            \begin{bmatrix}
                6 & 10\\
                10 & 24
            \end{bmatrix}
            \begin{bmatrix}
                -1-0\\
                4-5
            \end{bmatrix}=13.64\\
            \because T^{2} \sim \frac{( n-1 ) p} {( n-p )} F_{p, n-p}, \quad
            \therefore T^{2} \sim 3F_{2,2} \quad
            H_{0} \colon \mu^{\prime}= [0,5] \\
            \because \alpha=0.05 \therefore F_{2,2}(0.05)=19\\
            \because T^{2}=13.64<F_{2,2}(0.05)=57 \quad
            \therefore \alpha=0.05\text{时接受原假设} H_{0}
        \end{gathered}
        \]
\end{solution}

\newpage
\begin{problem}
    6.8.\\
    对三种处理收集了两种响应的观测值,观测值向量 $\begin{bmatrix} x_{1} \\ x_{2} \end{bmatrix}$ 为\\
    处理1:6 , 5 , 8 , 4 , 7\\
    处理2:3 , 1 , 2\\
    处理3:2 , 5 , 3 , 2
    \begin{enumerate}[label=(\alph*)]
        \item 对每个观测量做下述分解
        \[x_{ij}=\bar{x}+(\bar{x_i}-\bar{x_i})+(x_{ij}-\bar{x_i})\]
        观测值、总均值、处理效应、残差效应。
        \item 计算处理效应平方和\({SS}_{\text{处理}}\)与残差平方和\({SS}_{\text{残差}}\)。
    \end{enumerate}
\end{problem}

\begin{solution}
    \[
    \begin{gathered}
        \text{均值:}\frac{6+5+8+4+7+3+1+2+2+5+3+2}{12}=4\\
        \therefore  \begin{bmatrix}
            4 & 4 & 4 & 4 & 4 \\
            4 & 4 & 4 &  &  \\
            4 & 4 & 4 & 4 &
        \end{bmatrix}_{\text{均值}}\\
        \text{处理:}\frac{6+5+8+4+7}{5}=6=4+2,\quad \frac{3+1+2}{3}=2=4+(-2)\\ 
        \frac{2+5+3+2}{4}=3=4+(-1)
        \therefore   \begin{bmatrix}
            2 & 2 & 2 & 2 & 2 \\
            -2 & -2 & -2 &  &  \\
            -1 & -1 & -1 & -1 &
        \end{bmatrix}_{\text{处理}}\\
        \therefore 
        \begin{bmatrix}
            6-6 & 5-6 & 8-6 & 4-6 & 7-6 \\
            3-2 & 1-2 & 2-2 &  &  \\
            2-3 & 5-3 & 3-3 & 2-3 &
        \end{bmatrix}=
        \begin{bmatrix}
            0 & -1 & 2 & -2 & 1 \\
            1 & -1 & 0 &  &  \\
            -1 & 2 & 0 & -1 &
        \end{bmatrix}_{\text{残差}}
    \end{gathered}
    \]
    \[
    \begin{gathered}
        \text{对于变量1:}\\
        \begin{bmatrix}
            6 & 5 & 8 & 4 & 7 \\
            3 & 1 & 2 &  &  \\
            2 & 5 & 3 & 2 &  
        \end{bmatrix}_{\text{观测}}=
        \begin{bmatrix}
            4 & 4 & 4 & 4 & 4 \\
            4 & 4 & 4 &  &  \\
            4 & 4 & 4 & 4 &
        \end{bmatrix}_{\text{均值}}+
        \begin{bmatrix}
            2 & 2 & 2 & 2 & 2 \\
            -2 & -2 & -2 &  &  \\
            -1 & -1 & -1 & -1 &
        \end{bmatrix}_{\text{处理}}\\
        +
        \begin{bmatrix}
            0 & -1 & 2 & -2 & 1 \\
            1 & -1 & 0 &  &  \\
            -1 & 2 & 0 & -1 &
        \end{bmatrix}_{\text{残差}},\\
        \text{计算对应平方和:}
        {SS}_{\text{处理}}=5\times 2^2 + 3\times (-2)^2 +4 \times (-1)^4=36\\
        {SS}_{\text{残差}}=(-1)^2+2^2+(-2)^2+1^2+1^2+(-1)^2+(-1)^2+2^2+(-1)^2=18\\
        {SS}_\text{修正后总和}={SS}_{\text{观测}}-{SS}_{\text{均值}}=246-192=54\\
    \end{gathered}
    \]
\end{solution}

\begin{problem}
    例11.2【换数字】\\
    设组别\(\pi_1,\pi_2\)的概率密度函数分别是\(f_1(x)\)和\(f_2(x)\),又知\(c(1|2)=12\),\(c(2|1)=4\),根据以往经验给出\(p_1=0.6,p_2=0.4\)
    \begin{enumerate}[label=(\alph*)]
        \item 给出将一个心得观测值分入两个总体之一的最小ECM规则的一般形式。
        \item 对一个新的观测值产生的密度函数\(f_1(x_0)=0.36,f_2(x_0)=0.24\),根据以上信息将新项目分到哪个组别?
    \end{enumerate}
\end{problem}

\begin{solution}
    最小ECM判别规则为:
    \[
        \begin{cases}
            \text{若} \dfrac{f_1(x)}{f_2(x)} \geq \dfrac{c(1|2)p_2}{c(2|1)p_1}=\dfrac{12\times 0.4}{4 \times 0.6}=2 \quad \text{则} x \in \pi_1\\
            \text{若} \dfrac{f_2(x)}{f_2(x)} < \dfrac{c(1|2)p_2}{c(2|1)p_1}=\dfrac{12\times 0.4}{4 \times 0.6}= 2 \quad \text{则} x \in \pi_2
        \end{cases}
    \]
    对于新的观测值\(x_0\),有
    \[
        \frac{f_1(x_0)}{f_2(x_0)}=\frac{0.36}{0.24}=1.5<2,\quad \text{将新项目分到组别}\pi_2
    \]
\end{solution}


\begin{problem}
    确定下列协方差矩阵的总体主成分\(\zeta_1,\zeta_2,\zeta_3\)
    \[\Sigma=\begin{bmatrix} 5 & 2 & 0\\2&4&0\\0&0&2\end{bmatrix}\]
    并计算总体方差中第一主成分所解释的比例。
\end{problem}

\begin{solution}
    根据特征方程\(|\lambda I -E|=0\),有特征值和特征向量对为:
    \[ 
        \begin{cases}
            \lambda_1=6.562, & \quad e_1^\top=\begin{bmatrix} 0.788 & 0.615 & 0 \end{bmatrix}\\
            \lambda_2=2.438, & \quad e_2^\top=\begin{bmatrix} 0.615 & -0.788 & 0 \end{bmatrix}\\
            \lambda_3=2, & \quad e_3^\top=\begin{bmatrix} 0 & 0 & 1 \end{bmatrix}
        \end{cases}
    \]
    因此,主成分为
    \[
        \begin{cases}
            \zeta_1=0.788X_1+0.615X_2\\
            \zeta_2=0.615X_1-0.788X_2\\
            \zeta_3=X_3
        \end{cases}
    \]
    第一主成分所\(\zeta_1\)可以验证:
    \[
        \begin{aligned}
            Var(\zeta_1)&=Var(0.788X_1+0.615X_2)\\
            &=0.788^2Var(X_1)+0.615^2Var(X_2)+2\cdot 0.788\cdot 0.615Cov(X_1,X_2)\\
            &=0.788^2\cdot 5+0.615^2\cdot 4+2\cdot 0.788\cdot 0.615\cdot 2\\
            &=6.557
        \end{aligned}
    \]
    因此,第一主成分所解释的比例为:
    \[
        \frac{Var(\zeta_1)}{Var(X_1)+Var(X_2)+Var(X_3)}=\frac{6.56}{6.56+2.43+2}=59.65\%
    \]
\end{solution}

\begin{note}
    特征方程与特征向量:
    \[
        |\lambda I - \Sigma| = \left|\begin{matrix} \lambda-5 & -2 & 0 \\ -2 & \lambda-4 & 0 \\ 0 & 0 & \lambda-2 \end{matrix}\right| = 0
    \]
    \[
    \begin{gathered}
    (\lambda - 5)(\lambda - 4)(\lambda - 2) - (-2)^2(\lambda - 2) = 0\quad \text{此处直接敲计算器}\\
    (\lambda - 2)(\lambda^2 - 9\lambda + 16) = 0\\
    \lambda_{1,2} = \frac{-b \pm \sqrt{b^2 - 4ac}}{2a}= \frac{9 \pm \sqrt{81 - 64}}{2} = \frac{9 \pm \sqrt{17}}{2}\\
            \lambda_1 = \frac{9 + \sqrt{17}}{2} \approx 6.562,\,
            \lambda_2 = \frac{9 - \sqrt{17}}{2} \approx 2.438,\,
            \lambda_3 = 2
    \end{gathered}
    \]
    解线性方程组\((\lambda I - \Sigma)e = 0\)得到特征向量,如:
    \[
    \begin{bmatrix} \lambda_1 - 5 & -2 & 0 \\ -2 & \lambda_1 - 4 & 0 \\ 0 & 0 & \lambda_1 - 2 \end{bmatrix} = \begin{bmatrix} 1.562 & -2 & 0 \\ -2 & 2.562 & 0 \\ 0 & 0 & 4.562 \end{bmatrix}
    \]
    设特征向量\(e_1 = (x, y, z)^\top\),我们有:
    \[
    \begin{cases}
    1.562x - 2y &= 0 \\
    -2x + 2.562y &= 0 \\
    4.562z &= 0
    \end{cases}
    \]
    解得\(e_1 =(1,0.781,0)^\top=(0.788, 0.615, 0)^\top\),以此类推。
\end{note}




\end{document}